% Use the section* to for a more clear section break, assignments are not long documents they do not need an index 
\section*{Question Example}

A partition of a natural number $ n \ge 1 $ is any sequence of positive integers $ (a_1 , . . . , a_k ) $ such
that $ a_1 \ge a_2 \ge . . . \ge a_k $ and $ n = a_1 + a_2 + . . . + a_k $ . For example, partitions of $ n = 4 $ are
$ (1, 1, 1, 1) $, $ (2, 1, 1) $, $ (2, 2) $, $ (3, 1) $, $ (4) $. Numbers $ a_1 , . . . , a_k $ in a partition are called parts.



For natural numbers $ n, m \ge 1, m < n $, let $ A_{n,m} $ denote the collection of all partitions of $ n $ into
exactly $ m $ parts, and let $ B_{n,m} $ denote the collection of all partitions of $ n $ into at most $ m $ parts.
For example, $A_{4,2} = \{(2, 2), (3, 1)\} $ and $B_{4,2} = \{(2, 2), (3, 1), (4)\} $.


Task: Prove $ |A_{n,m}| = |B_{n-m,m}| $. In words, the number of partitions of a number n into
exactly $ m $ parts is equal to the number of partitions of $ n - m $ into at most $ m $ parts.


Hint: Observe that if $ n = a_1 + \dots + a_m $, then $ n - m = (a_1 -1) + \dots + (a_m - 1) $.

\begin{solution}

  % Function A -> B 
  
  Let $\phi$ denote the function $ \phi: A_{n,m} \rightarrow B_{n-m,m}$, defined as $\phi = g_1f_1$,

  Where $f_1$ is a function defined as $$ f_1((a_1 + \dots + a_m)) = ((a_1 - 1) + \dots + (a_m - 1)) $$ 

  And $g_1$ is a function defined as $$ g_1((a_1 + \dots + a_m)) = (a_1 + \dots + a_k)\;\; \text{where $ k = m - |\:\{ a_i \in \{a_1,\dots,a_m\} \;|\; a_i = 0 \}\:|$} $$

  In words, $\phi$ is a function that subtracts 1 from each integer in the sequence and then removes all 0's from the sequence by slicing off the end of the sequence, 
  the amount sliced off is equal to the difference between m and the number of 0's in the sequence. This is done to make sure the sequence is stil an integer partition.

  Note that, $n = a_1 + \dots + a_m \implies n - m = ((a_1 - 1) + \dots (a_m - 1))$\\

  % Function B -> A 
  Let $\beta_m$ denote the function $ \beta_m: B_{n-m,m} \rightarrow A_{n,m}$ defined as,

  $$ \beta_m((a_1 + \dots + a_k)) = ((a_1 + 1_1) + \dots + (a_k+1_k) + \dots + 1_m) $$
  
  In words, $\beta_m$ merges the sequence $ 1_1 + \dots + 1_m $ to the input sequence. Note
  that the image will always have exactly $m$ parts and that any element in $B_{n-m,m}$ will 
  result in one distinct element of $A_{n,m}$.

  % Inverse 
  Now, $|A_{n,m}| = |B_{n-m,m}|$ if and only if $\beta_m\phi = e$.

  \begin{align*}
    \beta_m\phi((a_1 +\dots+ a_m)) = \beta_m(((a_1-1) + \dots + (a_k - 1))) \\    
    = ((a_1) +\dots+ (a_k) + \dots + 1_m) \\
    = a_1 + \dots + a_m \\
  \end{align*}

  Therefore it is clear to see that for any sequence in $A_{n,m}$, $\;\beta_m\phi = e$. Therefore $\beta_m$ is an inverse of $\phi$ therefore $\phi$ is bijective. Therefore $ |A_{n,m}| = |B_{n-m,m}|$. 


\end{solution}
 
